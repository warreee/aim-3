%----------------------------------------------------------------------------------------
%	PACKAGES AND OTHER DOCUMENT CONFIGURATIONS
%----------------------------------------------------------------------------------------

\documentclass[11pt,a4paper]{article}

\usepackage[utf8]{inputenc}
\usepackage[english]{babel}
\usepackage{amsmath}
\usepackage{amsfonts}
\usepackage{amssymb}
\usepackage[left=2cm,right=2cm,top=2cm,bottom=2cm]{geometry}
\usepackage{graphicx}
\usepackage{float}
\graphicspath{{figures/}}
\usepackage{todonotes}
\usepackage{url}
\usepackage{subcaption}
%----------------------------------------------------------------------------------------
%	CONFIGURATION OF CODE LISTINGS
%----------------------------------------------------------------------------------------

\usepackage{listings}
\usepackage{color}
 
\definecolor{codegreen}{rgb}{0,0.6,0}
\definecolor{codegray}{rgb}{0.5,0.5,0.5}
\definecolor{codepurple}{rgb}{0.58,0,0.82}
\definecolor{backcolour}{rgb}{0.95,0.95,0.92}
 
\lstdefinestyle{mystyle}{
    backgroundcolor=\color{backcolour},   
    commentstyle=\color{codegreen},
    keywordstyle=\color{magenta},
    numberstyle=\tiny\color{codegray},
    stringstyle=\color{codepurple},
    basicstyle=\footnotesize,
    breakatwhitespace=false,         
    breaklines=true,                 
    captionpos=b,                    
    keepspaces=true,                 
    numbers=left,                    
    numbersep=5pt,                  
    showspaces=false,                
    showstringspaces=false,
    showtabs=false,                  
    tabsize=2
}
 
\lstset{style=mystyle}

\setcounter{section}{1}

\begin{document}

\begin{titlepage}

\newcommand{\HRule}{\rule{\linewidth}{0.5mm}} % Defines a new command for the horizontal lines, change thickness here

\center % Center everything on the page
 
%----------------------------------------------------------------------------------------
%	HEADING SECTIONS
%----------------------------------------------------------------------------------------

\textsc{\textsc{\LARGE TU Berlin}}\\[1.5cm] % Name of your university/college
\textsc{\Large Advanced Information Management}\\[0.5cm] % Major heading such as course name
\textsc{\large Homework assignment 1}\\[0.5cm] % Minor heading such as course title

%----------------------------------------------------------------------------------------
%	TITLE SECTION
%----------------------------------------------------------------------------------------

\HRule \\[0.4cm]
{ \huge \bfseries Programming in Hadoop and Clustering Excercises}\\[0.4cm] % Title of your document
\HRule \\[1.5cm]
 
%----------------------------------------------------------------------------------------
%	AUTHOR SECTION
%----------------------------------------------------------------------------------------

\begin{minipage}{0.4\textwidth}
\begin{flushleft} \large
\emph{Author:}\\
Ward \textsc{Schodts} % Your name
\end{flushleft}
\end{minipage}
~
\begin{minipage}{0.4\textwidth}
\begin{flushright} \large
\emph{Supervisor:} \\
Juan \textsc{Soto} \\ % Supervisor's Name

\end{flushright}
\end{minipage}\\[4cm]

% If you don't want a supervisor, uncomment the two lines below and remove the section above
%\Large \emph{Author:}\\
%John \textsc{Smith}\\[3cm] % Your name

%----------------------------------------------------------------------------------------
%	DATE SECTION
%----------------------------------------------------------------------------------------

{\large \today}\\[3cm] % Date, change the \today to a set date if you want to be precise

%----------------------------------------------------------------------------------------
%	LOGO SECTION
%----------------------------------------------------------------------------------------

%\includegraphics[scale=0.15]{sedes}\\[1cm] % Include a department/university logo - this will require the graphicx package
\begin{figure}
    
    \def\svgwidth{\columnwidth}
    \scalebox{.2}{
    \input{figures/logo.pdf_tex}}
\end{figure}
 
%----------------------------------------------------------------------------------------

\vfill % Fill the rest of the page with whitespace

\end{titlepage}

\section*{Programming in Hadoop}
\subsection*{1. WordCount - "Hello World" of MapReduce}
\lstinputlisting[language=Java, caption=FilteringWordCount.java]{../HadoopTask/src/main/java/de/tuberlin/dima/aim3/assignment1/FilteringWordCount.java}

\subsection*{2. A custom Writable}
\lstinputlisting[language=Java, caption=PrimeNumbersWritable.java]{../HadoopTask/src/main/java/de/tuberlin/dima/aim3/assignment1/PrimeNumbersWritable.java}

\subsection*{3. Average temperature per month}
\lstinputlisting[language=Java, caption=AverageTemperaturePerMonth.java]{../HadoopTask/src/main/java/de/tuberlin/dima/aim3/assignment1/AverageTemperaturePerMonth.java}

\pagebreak
\section*{Clustering}
\subsection*{1. Metrics I}
We have 3 points A(4,8), B(9,5) and C(2,2).\\
The \textit{centroid} is then:
$$ (x,y) = \left(\frac{4+9+2}{3},\frac{8+5+2}{3}\right)= (5,5)$$
We now calculate the error towards the centroid for every point:\\
\\
A(4,8):

$$\sqrt{(4-5)^2 + (8-5)^2} = \sqrt{10}$$
\noindent
B(9,5):

$$\sqrt{(9-5)^2 + (5-5)^2} = \sqrt{16} = 4$$
\noindent
C(2,2):

$$\sqrt{(2-5)^2 + (2-5)^2} = \sqrt{18} = 3\cdot\sqrt{2}$$

\noindent
The SSE is then:
$$\sqrt{10}^2 + 4^2 + 3\cdot\sqrt{2}^2 = 44$$

\subsection*{2. Metrics II}
If you partition 3 points into 2 clusters, there's one with one element and one with two.
The SSE of 1 point is always 0. So the total SSE is only determined by the SSE of the possible combinations of two points.
\\
\\
$$Centroid((3,0),(0,7))=(1.5,3.5)$$
$$SSE((3,0),(0,7))=\sqrt{2.25+12.25}^2 + \sqrt{2.25+12.25}^2 = 29$$
$$Centroid(((0,7),(6,5))=(3,6)$$
$$SSE(((0,7),(6,5))=\sqrt{9+1}^2 + \sqrt{9+1}^2 = 20$$
$$Centroid((3,0),(6,5))=(4.5,2.5)$$
$$SSE((3,0),(6,5))=\sqrt{2.25+6.25}^2 + \sqrt{2.25+6.25}^2 = 17$$

The SSE is the smallest in the last configuration therefore the the most optimal split in two groups is the following:
\\
\\
Cluster 1: [(0,7)]\\
Cluster 2: [(3,0),(6,5)]
\end{document}